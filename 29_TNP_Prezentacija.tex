\documentclass{beamer}
\usepackage{beamerthemeshadow}
\usepackage{graphicx}
\usepackage{color}
\usepackage[utf8]{inputenc}
\usepackage{hyperref}
\usepackage[flushleft]{threeparttable}
\definecolor{beamer@darkred}{rgb}{0.85,0.1,0.1}
\setbeamercolor{structure}{fg=beamer@darkred}

\def\d{{\fontencoding{T1}\selectfont\dj}}
\def\D{{\fontencoding{T1}\selectfont\DJ}}


\title{Tehničko i naučno pisanje}
\subtitle{-- Metodologija stručnih i naučnih izlaganja --}
\author{dr Jelena Graovac}
\institute{Matematički fakultet\\Univerzitet u Beogradu}
\date{
	\footnotesize{Beograd, 2020.}	
}

\begin{document}
\begin{frame}
	\thispagestyle{empty}
	\titlepage
\end{frame}

\addtocounter{framenumber}{-1}

\begin{frame}[fragile]\frametitle{Literatura}
	\begin{itemize}
		\item Zasnovano na:\\
		Goran Nenadic, Predrag Janičić, Aleksandar Samardžić: \LaTeX{} za autore, Beograd, Kompjuter biblioteka, 2003.
		(\url{http://poincare.matf.bg.ac.rs/~janicic//latex2e/})
	\end{itemize}
\end{frame}

\begin{frame}
	\frametitle{Pregled} % Table of contents slide, comment this block out to remove it
	\tableofcontents[hidesubsections] 
\end{frame}

\section{Metodologija stručnih i naučnih izlaganja}

\begin{frame}[fragile]\frametitle{Kako ljudi pamte?}
	\begin{itemize}	
		\item 10\% onoga što pročitaju
		\item 20\% onoga što čuju
		\item 30\% onoga što vide
		\item 50\% onoga što čuju i vide
		\item 70\% onoga što kažu i napišu
		\item 90\% onoga što rade
	\end{itemize}
\end{frame}

\subsection{Kako ljudi uče?}

\begin{frame}[fragile]\frametitle{Kako ljudi uče?}
	\begin{itemize}	
		\item Verbalnim putem (čitanjem i slušanjem)
		\item Vizuelnim putem (gledanjem)
		\item Aktivnim učešćem (gledanjem, slušanjem i činjenjem)
	\end{itemize}
\end{frame}

\section{Zakljucak}

\begin{frame}[fragile]\frametitle{Kako ljudi uče?}
	\begin{itemize}	
		\item Ljudi ne komuniciraju samo rečima. Prema nekim
		istraživanjima:
		\begin{itemize}
			\item samo 8\% poruke se prenese samim rečima (verbalna komunikacija)
			\item 37\% se prenesi bojom glasa, tonalitetom, pauzama u govoru (paralingvističkim znakovima)
			\item 55\% poruke se prenosi govorom tela: pratećim pokretima, izrazom lica i
			očiju, stavom tela i drugo (neverbalna komunikacija)
		\end{itemize}
		\item Verbalnim putem se najčešće prenose činjenice i
		sirove informacije, dok se neverbalnim putem prenose stavovi i
		emocionalni odnos prema činjenicama koje izlažemo.
	\end{itemize}
\end{frame}

\end{document}